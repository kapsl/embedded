\section{Einführung}
Roomba Kart ist ein an Mario Kart angelegtes Rennspiel, welches von zwei Spielern gegeneinander gespielt werden kann. Die Karts sind zwei umfunktionierte Staubsauger Roboter des Herstellers Roomba. Die Karts werden von zwei Logitech Fernbedienungen gesteuert. Auf dem Arduino Board ist ein Funkmodul angebracht, welches eine Kommunikation zwischen den Roombas erlaubt. Hierdurch können sich die Spieler via vorher erworbener Powerups beeinträchtigen oder vor Attacken schützen.  Ein Powerup kann durch das Dichte vorbeifahren an einem vorher definierten Gegenstand oder am Gegener erworben werden. Folgende Powerups stehen zur Verfügung: 

\begin{itemize}
\item RED TANK: Wird als kleiner Panzer auf dem Display des Karts ausgegeben.\\Wirkung: Wenn dieses Powerup abgeschossen wird, dreht sich das gegnerische Kart um $1080^\circ$. Ein verfehlen ist nur dann möglich 	wenn sich der gegnerische Spieler rechtzeitig durch einen BIG DADY schützen 
\item BIG DADY: Wird als großer Panzer auf dem Display des Karts ausgegeben.
\\Wirkung: Wenn dieses Powerup aktiviert wird, besteht ein 7 Sekunden Schutz vor  
dem Powerup RED TANK. 
\item MUSHROOM: Wird als Pilz auf dem Display des Karts ausgeben. 
\\Wirkung: Kart kann 7 Sekunden ungestraft die Seitenbegrenzung der Strecke verlassen. 		
Wenn der Kurs verlassen wurde aber nicht rechtzeitig wieder befahren wird hat dieser 	
Spieler das Rennen verloren und das Kart bleibt stehen.
\end{itemize}

\section{Spielstart}
Die mindeste und maximale Anzahl von Spielern, die beim Roomba Kart mitmachen können, ist zwei. Um mitspielen zu können, braucht jeder Spieler einen Roomba, ein mit dem Spiel programmiertes Arduino Board mit Funkmodul und eine Ferbedienung. Das Arduino Board wird durch eine Steckverbindung mit dem Roomba verbunden. Im Anschluss sollten auf dem Roomba-Display vier Bindestriche anzeigt werden. Falls dies nicht der Fall ist, muss der Resetbutton auf dem Arduino Board gedrückt werden.

Durch die vier Bindestriche werden die Spieler aufgefordert anzugeben, ob sie Spieler eins oder Spieler zwei sind. Durch die Eingabe der entsprechenden Ziffer durch die Fernbedienung und einer Bestätigung der Eingabe durch die Rechtstaste gefolgt von dem Powertaste, werden die Spieler festgelegt. Die Spieler müssen sich unterscheiden. Bei der gleichen Eingabe muss ein Roomba durch pressen des Resetbuttons auf dem Arduino Board zurückgesetzt werden und die Spielerwahl muss erneut durchgeführt werden. 
   
\section{Bedienung}
Im folgenden werden kurz die allgemeine Steuereigenschaften erklärt. Danach wird die Steuerung für Spieler 1 und im Anschluss die Steuerung für Spieler 2 erklärt. 

\underline{Allgemein:} 

Das Kart hält selbständig seine Geschwindigkeit. Die Spieler können durch die Vorwärtstaste bzw. Rückwärtstaste einfluss auf die Geschwindigkeit sowie die Fahrtrichtung (Vorwärts/Rückwärts) nehmen. Durch die Rechts- und Linkstaste bestimmt der Spieler in welche Richtung sein Kart fahren soll. Desto länger die Links- oder Rechtstaste gedrückt wird desto stärker lenkt das Kart in die jeweilige Richtung. Wenn Links- oder Rechtstaste nicht mehr gedrückt wird, fährt das Kart wieder von selbstständig geradeaus. Eingesammelte Powerups können durch die Schusstaste abgeschossen.

\underline{Spieler 1:}

Spieler 1 steuert sein Kart mit dem Steuerkreuz der Fernbedienung und feuert seine Powerups mit der Powertaste ab. Eine detaillierte Beschreibung bietet die folgende Tabelle:

\vspace{0.5cm}
\begin{tabular}{|l|l|}
\hline
Taste & Auswirkung \\ \hline
Steuerkreuz hoch & Kart fährt vorwärts \\ \hline
Steuerkreuz runter & Kart fährt rückwärts/bremst \\ \hline
Steuerkreuz links & Kart fährt nach links \\ \hline
Steuerkreuz rechts & Kart fährt nach rechts \\ \hline
Powertaste & Kart feuert Powerup ab, falls vorhanden \\ \hline  
\end{tabular} 
\vspace{0.5cm}

\underline{Spieler 2:}
 
Spieler 2 steuert sein Kart mit dem Tastenfeld der Fernbedienung und feuert seine Powerups mit der Taste 5 ab. Eine detaillierte Beschreibung bietet die folgende Tabelle:

\vspace{0.5cm}
\begin{tabular}{|l|l|}
\hline
Taste & Auswirkung \\ \hline
Taste 2 & Kart fährt vorwärts \\ \hline
Taste 8 & Kart fährt rückwärts/bremst \\ \hline
Taste 4 & Kart fährt nach links \\ \hline
Taste 6 & Kart fährt nach rechts \\ \hline
Taste 5 & Kart feuert Powerup ab, falls vorhanden \\ \hline 
\end{tabular}
\vspace{0.5cm}

\section{Regelwerk}

 



































